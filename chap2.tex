\chapter{گام دوم}\label{chap:2}
\section{بررسی ‌الگوهای مولفه اتصال مختلف در سامانه‌ی ما}

\زیرقسمت{الگوی ‌‌Broker}

الگوی ‌‌Broker سخن از این نکته می‌گوید که در سامانه‌هایی که سیستم مجموعه‌ای از سرویس‌های توزیع‌شده است  پیاده‌سازی مشکل خواهد بود و ارتباط  اجزای سیستم و در دسترس بودن سرویس‌های کامپوننت‌های مختلف برای ما مشکل‌ساز خواهد شد. لزوم اطلاع از مکان و نوع سرویس‌ها از سوی  کاربران  برای استفاده از سرویس‌ها باعث ایجاد و تقویت مشکلات ذکر شده خواهد شد. راه حلی که این الگو پیشنهاد میدهد استفاده از یک میانجی است که در مورد سویس‌ها اطلاع دارد و کاربران و سرویس‌دهندگان به جای اتصال به یکدیگر به آن متصل می‌شوند. با توجه به اینکه سامانه ما یک سیستم توزیع شده است که سرویس‌هایی در آن ارائه می‌شود، قاعدتا با مشکلات گفته شده مواجه خواهیم شد لذا راه حلی که این الگو  ارائه داده است  برای سامانه‌ای مشابه سامانه ما مناسب به نظر می‌رسد.



\زیرقسمت{الگوی ‌‌Model-View-Controller}

مسئله‌ای که این الگو به آن توجه کرده است تغییرات مداوم واسط کاربری است، بدین معنا که کاربران اطلاعات مختلف را به اشکال مختلف خواهند دید و با آن تعامل خواهند داشت و لذا بسیار  مفید خواهد بود اگر این واسط در عین اینکه  حالت فعلی اطلاعات را به کاربران نشان می‌دهد عملکردی مستقل از عملکرد کاربردی سیستم داشته باشد. راه حلی که این الگو ارائه داده است تقسیم عملکرد سیستم بین سه مولفه model ، view و controller است. مولفه view وظیفه‌ی نمایش بخش‌هایی از داده و تعامل با کاربران را بر عهده دارد، model شامل داده‌های سامانه می‌شود و controller میانجی بین این دو است و مدیریت اخطارهای تغییر حالت را بر عهده دارد. به طور کلی در شبکه‌های اجتماعی و سامانه‌های پیام‌رسان با توجه به اینکه تعداد کاربران هم بسیار بالا است و سیستم بایستی با کاربران تعامل داشته باشد واسط‌کاربری و تغییرات آن برای ما حیاتی خواهد شد.علاوه بر این جذابیت چنین سامانه‌ای برای جذب کاربران اهمیت دارد لذا اگر امکانات شخصی سازی در  واسط‌کاربری به کاربران داده شود تاثیر خوبی خواهد داشت. با توجه به همه‌ی این نکات الگوی MVC برای سامانه‌ی ما مناسب به نظر می‌رسد.

