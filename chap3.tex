

\زیرقسمت{الگوی \متن‌لاتین{Service Oriented Architecture }}
موقعیتی که در بکارگیری این الگو بکار می‌رود زمانی است که تعدادی سروریس توسط فراهم کنندگان سرویس ارائه شده است و تعدادی مصرف کننده آنها را مصرف می‌کنند و لازم است مصرف کنندگان لازم است این سرویس‌ها را درک کنند و بتوانند از آنها استفاده کنند بدون آنکه از جزئیات پیاده‌سازی سرویس‌ها با خبر باشند. در سامانه‌ی پیام رسان مصرف کنندگان و فراهم کنندگان سرویس را می‌توان قطعات مختلف در نظر گرفت. مسئله‌ای که این الگو حل می‌کند همکاری بین قطعات مختلف در زمان اجرا است که می‌توانند با زبانهای برنامه نویسی مختلفی نوشته شده باشند. در بررسی سامانهای پیام‌رسان مشاهده شد که اجزاء مختلف می‌توانند زبانها و platform های مختلف داشته باشند.\مرجع{TheWhats:online}\مرجع{ScalingT:online} \\
همانطور که در قسمت \رجوع{sec:soa}  به طور مفصل درباره‌ی بکارگیری این الگو توضیح داده شد، این الگو در سامانه‌ی پیام‌ رسان می‌تواند نقش اساسی را ایفا کند.

\زیرقسمت{الگوی Publish-Subscribe}
این الگو زمانی بکار برده می‌شود که تعداد زیادی تولید کننده و مصرف کننده داده وجود دارد که لازم است به صورت مستقل تعامل برقرار کنند. تعداد این تولید کنندگان و مصرف کنندگان نامشخص و غیر ثابت است. چنین شرایطی را در سامانه‌ی پیام رسان نیز داریم. تعداد کاربران زیاد است، نامشخص و متغیر و مستقل از هم هستند و هر یک تولید کننده و مصرف کننده‌ی مستقل هستند. مسئله‌ای که این الگو حل می‌کند وجود یک مکانیزم مجتمع برای انتقال اطلاعات بین این اجزاء است که آنها از هویت یکدیگر یا حتی از وجود هم بی خیر هستند. \\
یک نمونه از مورد کاربردی که این الگو در سامانه‌ی پیام رسان دارد می‌تواند داشته باشد این است که کاربران وضعیت خود را به روز رسانی کنند یا رویدادی را به اطلاع سایرین برسانند. به عنوان مثال آنلاین یا آفلاین بودن خود را با دیگر کاربران به اشتراک بگذارند.

\زیرقسمت{الگوی Shared-Data}
در هنگامی که قطعات  \پانویس{Component} محاسباتی مختلف نیازمند آن هستند که حجم بزرگی از داده‌ها را به اشتراک بگذارند و این داده‌ها به قطعه‌ی خاصی تعلق ندارد. در سامانه‌ی پیام‌ رسان نیز حجم داده‌ها فراوان است. این الگو مسئله‌ای که حل می‌کند عبارت است از ذخیره‌سازی و تغییر داده‌های پایدار در سامانه که این داده‌ها توسط چندین قطعه مورد دسترسی قرار می‌گیرند. در جایی دیگر بیان شده که ارتباط بین قطعات به واسطه‌ی مخزن داده‌ی مشترک  \پانویس{Shared Data Store}انجام می‌گیرد. در صورتی که در سامانه‌ی پیام رسان قطعات راههای ارتباطی دیگری نیز دارند. همچنین در توضیحات این الگو مشخص شده است که کدام قطعات به کدام مخازن متصل هستند که این رابطه نیز می‌تواند در سامانه‌ی پیام رسان وجود داشته باشد. به طور کلی اگر پیاده‌سازی این الگو را در قالب پایگاه داده‌های تجاری در نظر بگیریم برای سامانه‌ی پیام رسان مناسب است و در غیر این صورت مناسب نیست.







