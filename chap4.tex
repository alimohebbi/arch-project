\section{ملاحظات فنی}
\زیرقسمت{ملاحظات محیط توسعه}
\begin{itemize}
\فقره از سامانه‌های کنترل نسخه استفاده شود. یک گزینه‌ی مناسب استفاده از GitLab می‌باشد. در آن می‌توان سرویس‌های مختلف را در قالب پروژه‌هایی در نظر گرفت و برنامه نویسان به پروژه سطوح دسترسی مختلف داشته باشند. 
\فقره از سامانه‌های ردگیری مشکلات به منظور اختصاص وظایف استفاده کرد. سامانه‌ی پیشنهادی jira می‌باشد.
\فقره از سامانه‌های ادغام مداوم (CI)  \پانویس{continuous integration} به منظور راحتی ادغام و استقرار استفاده شود. 
\end{itemize}
\زیرقسمت{ملاحظات پیاده‌سازی}
\begin{itemize}


\فقره در ذخیره سازی اطلاعات مربوط به کاربران  و فراداده‌ها از پایگاه داده‌ی رابطه‌ای  PostgreSQL استفاده شود و در ذخیره سازی داده‌هایی که Scheme جدول آنها ممکن است زیاد تغییر کند یا داده‌های چند رسانه‌ای، از پایگاه داده NoSQL ، MongoDB استفاده شود. 
\فقره از زبان برنامه نویسی Java استفاده شود. این زبان قابلیتهای فراوانی دارد. برای ساخت پروژه‌های وسیع مناسب است. نیروی متخصص آن به سادگی یافت می‌شود. 
\فقره از سیستم عامل Linux استفاده شود. این سیستم عامل از کارایی و امنیت بالاتری نسبت به ویندوز دارد. 
\فقره برای پیاده‌ سازی قابلیت Cache از نرم افزار  Memcached استفاده شود.
\end{itemize}