\chapter{گام سوم}

\قسمت{ارزیابی}
\زیرقسمت{ توضیح کلی}

سعی داریم تا با ارزیابی سبک وزن معماری از مزایای ارزیابی بهره‌مند شویم و در عین حال با مشکلات و هزینه‌های ارزیابی کامل روبرو نشویم. برای همین تصمیم داریم که جلساتی که انجام آن‌ها 20 تا 30 نفر روز زمان نیاز داشت را در نصف روز الی یک روز انجام دهیم، که به طور کلی اعضای درون سازمانمان در آن شرکت خواهند کرد، و تبعا به حد ارزیابی سنگین وضع وارد عمق و جزئیات معماری نخواهیم شد. با توجه به اینکه افراد حاضر در ارزیابی درون سازمانی هستند درک قبلی ایشن از معماری خود باعث کاهش زمان مصرفی اولیه جهت فهم موضوع خواهد شد.

\زیرقسمت {افراد حاضر در جلسه}

\begin{itemize}

	

	\فقره ارزیابی کنندگان: چندین نفر از اعضای درون سازمان را به عنوان اعضای تیم ارزیابی مشخص می‌کنیم.

	

	\فقره مدیران پروژه: کسانی که در مورد پروژه تصمیم می‌گیرند و یا تامین نیازهای مالی پروژه را به عهده دارند در این گروه می‌باشند.

	

	\فقره سایر افراد : هر کسی که در پروژه نقشی دارد و در دو گروه بالا جای نداشت، تحت عضویت این گروه حاضر خواهد شد.

	

\end{itemize}

بدیهی است که  در این‌جا بازیگر همه‌ی این نقش ها اعضای این گروه، علی محبی و مهدی کشانی هستند.

\زیرقسمت {چگونگی اجرای جلسات}

\begin{itemize}

	

	\فقره در ابتدای کار یک توضیح خیلی سریع و کلی در مورد محرک‌های کسب و کار پچروژه بیان خواهد شد تا اطمینان حاصل شود که نکات مورد نیاز به همه یادآوری شود.

	

	\فقره یک مرور کلی از معماری سیستم ارائه خواهد شد و در آن از برخی دید‌هایی که در معاری تولید شده اند استفاده خواهد شد.

	

	\فقره رویکردی که معماری در قبال هر یک از ویژگی‌های کیفی سامانه در پیش گرفته است توضیح داده خواهد شد.

	

	\فقره سناریوهای موجود را در قالب یک درخت مثل درخت سودمندی ارائه خواهند شد.



	\فقره در این مرحله ارزیابی رویکردهای معماری را خواهیم داشت و سناریوهایی که امتیاز بالایی می‌گیرند را به معماری نگاشت می‌کنیم.این مرحله بیشترین زمان را در میان سایر مراحل به خود اختصاص می‌دهد.

	\فقره در آخر ریسک‌ها، غیر ریسک‌ها،سناریوهاو نقاط موازنه جدید و قدیمی سیستم بیان می‌شوند و در مورد اینکه آیا ریسک جدیدی نیز اضافه شده است یا خیر بحث می‌شود.

	 

\end{itemize}

\section{لیست ریسک‌های عمومی}



\زیرقسمت {ریسک‌های مربوط به حجم پروژه:} 

\begin{itemize}

	

	\فقره حجم تخمین زده شده و تعداد خطوط کد چقدر خواهد بود؟

	

	\فقره چقدر این تخمین قابل اطمینان است؟

	

	\فقره حجم پایگاه‌داده‌ای که توسط سامانه تولید و یا استفاده می‌شود.

	\فقره تعداد کاربرانی که از سامانه استفاده می‌کنند.

	

\end{itemize}



\زیرقسمت {ریسک‌های مربوط به تگنولوژی‌های مورد استفاده:} 

\begin{itemize}

	

	\فقره آیا تکنولوژی که مورد استفاده قرار می‌گیرد برای شرکت ما جدید است؟

	

	\فقره آیا مولفه‌های درون این سامانه قبلا توسط ما در پروژه‌های دیگر پیاده شنده بوده است؟

	

	\فقره آیا نیازمندی‌ها مارا مجبور به استفاده از تکنولوژی‌ها و الگوریتم‌های جدیدی می‌کنند؟

	\فقره آیا برای پایگاه داده‌ای که می‌خواهیم استفاده کنیم در چنین زمینه‌هایی مشابه کار ما امتحان خود را 

	پس داده‌اند؟

	

\end{itemize}



\زیرقسمت {ریسک‌های دیگر:} 

\begin{itemize}

	

	\فقره آیا افراد حاضر مهارت‌های مورد نیاز را برای تعابیری که در معماری اندیشیده ایم دارند؟

	

	\فقره یا توجه به اینکه سامانه ما یک سامانه پیام‌رسان است چقدر برای کاربران جذاب و فریبنده است؟

	

	\فقره آیا ملاحظات مربوط به جامعه و حکومت در سامانه ما رعایت شده است؟

	\فقره پیش‌بنی اثراتی که جامعه، محیط و حکومت بر سامانه ما خواهند داشت.

	

	\فقره اگر در زمانی که مد نظرمان بوده پروژه تمام نشود چه خواهد شد؟

	

\end{itemize}

\section{لیست ریسک‌های اختصاصی}





\begin{itemize}

	

	\فقره الگوی pub/sub استفاده کردیم این الگو می‌تواند اثر تاخیر را افزایش دهد و اثر منفی بر روی قابلیت پیش‌بینی زمان تحویل پیام‌ها داشته باشد.

	

	\فقره همینطور اثر منفی این الگو بر روی عدم اطمینان از تحویل و عدم ترتیب پیام‌ها قابل مشاهده است.

	

	\فقره همچنین استفاده از SOA باعث افزایش پیچیدگی سیسستم می‌شود.

	\فقره سربارهای استفاده از میانجی‌ها گریبان‌گیر کارایی سامانه ما خواهد شد.

	

	\فقره سرویس‌ها ممکن است که گلوگاه سیستم شوند و کارایی را تحت تاثیر قرار دهند.

	

	\فقره استفاده از مخزن ذخیره داده نیز امکان گلوگاه کارا یی شدن را دارد.

	\فقره مخزن ذخیره داده امکان از کار انداختن کل سیستم در صورت خرابی را به وجود می‌آورد.

	\فقره تولد و مصرف کنندگان داده ممکن است به شدت به یکدیگر وابسطه شوند.

	\فقره استفاده از broker ارتباط غیر مستقیم را به وجود می‌آورد که این تاخیر ایجاد کند، یا گلوگاه شود.

	\فقره خراب شدن broker ممکن است خرابی کلی به وجود بیاورد.

	\فقره ‌broker ممکن است هدف حملات امنیتی قرار گیرد.

	\فقره broker ممکن است به سختی قابل آزمایش باشد.

\end{itemize}



\section{نقاط موازنه}

با توجه به نکاتی که در قسمت قبل گفته شد که استفاده از یک سری الگوها شرایط حادی ممکن است برای ما به وجود بیاورد. در چندین جا مشاهده شد که یک خرابی باعث خرابی کل سیستم شود یا اینکه گلوگاه‌های کارایی به وجود بیاید. با توجه به اینکه کاربرد هر یک از این الگوها قبلا گفته شده است حال باید ببینیم با توجه به این اثراتی که بر سیستم ما دارند، آیا بازهم عاقلانه است از آن‌ها استفاده کنیم؟





\زیرقسمت {لیست نقاط موازنه:}

 

\begin{itemize}

	\فقره با توجه به اینکه الگوی Pub/Sub می‌تواند اثرات منفی بر زمان تحویل یا ترتیب و تضمین تحویل پیام‌ها داشته باشد در چه حد استفاده از آن برای سامانه ما مناسب است؟ آیا می‌توان آن را تا حدی تنظبم کرد که هم از مزایایش بهره‌مند شویم هم معایبش گربان‌گیرمان نشود؟ یا به طور کلی بایستی کنار گذاشته شود؟

	\فقره آیا استفاده از الگوی SOA با توجه به اثراتی که بر روی کارایی و پیچیدگی دارد که در قسمت قبل مشاهده کردیم آیا استفاده از آن برای سامانه ما مناسب است؟ در چه حد استفاده از این الگو برای ما کفایت می‌کند؟ آیا می‌توانیم آن را در حد نیاز خود شخصی سازی کنیم؟

	

	\فقره اثرات مخرب حافظه اشتراکی را در بالا دیدیم. آیا می‌توان اثرات آن را خنثی کرد؟ مثلا با استفاده از حافظه‌های پشتیبان؟ برای استفاده از مزایای این الگو با متحمل شدن کمترین ریسک‌ها جه پیشنهادی هست؟

	\فقره آیا اثرات منفی استفاده از الگوی beoker که در قسمت قبل دیدیم جبران شدنی هستند؟ یا قابلیت تنظیم دارند؟ در نهایت این الگو را مورد استفاده قرار بدهیم یا خیر با موازنه کردن عیب‌ها و مزایایش

	

	

\end{itemize}





\قسمت{نیازمندی‌های کیفی}

\زیرزیرقسمت{کارایی}
\begin{itemize}
\فقره منبع : کاربران نهایی
\فقره محرک : کاربران شروع به گفتگو در یک گروه می‌کنند
\فقره محصول : کل سامانه
\فقره محیط : سامانه در ساعات پیک ترافیک است
\فقره پاسخ : استقرارهای جدیدی از سرویس ها برقرار می‌شود
\فقره معیار پاسخ : دریافت متن‌های جدید در گروه در کمتر از .۵ انجام خواهد شد
\فقره تصمیم معماری : با توجه به استفاده از معماری SOA سرویس های جدید  مرطبت با چت گروهی استقرار می‌یابند و متعادل کننده بار ، بار پردازشی و ترافیکی را در میان سرویس ها تقسیم می‌کنند.
\end{itemize}

\زیرزیرقسمت{کارایی}
\begin{itemize}
\فقره منبع : کاربران نهایی
\فقره محرک : کاربر قصد دارد از حالت آنلاین خارج شود
\فقره محصول : مولفه‌های session ، connection ، pubsun
\فقره محیط : سامانه در عملکرد کاری معمولی است
\فقره پاسخ : وضعیت جدید کاربر به افراد موجود در لیست او اطلاع داده می‌شود 
\فقره معیار پاسخ :وضعیت جدید کاربر در حداکثر ۳ ثانیه منتشر می‌شود
\فقره تصمیم معماری : استفاده از الگوی pubsub موجب می‌شود که انتشار رویدادها با سرعت زیاد و با کمترین سربار انتشار یابد
\end{itemize}


\زیرزیرقسمت{دسترسی}
\begin{itemize}
\فقره منبع : یکی از سرویس ها
\فقره محرک : خطایی رخ می‌دهد و باعث شکست سرویس نمی‌شود
\فقره محصول : سرویس مربوطه 
\فقره محیط : سامانه در عملکرد کاری معمولی است
\فقره پاسخ : وضعیت به مسوولین سامانه گزارش داده می‌شود
\فقره معیار پاسخ : خطا در کمتر از $/2 $   ثانیه تشخیص داده می‌شود و در کمتر از  $/02 $ ثانیه اطلاع داده می‌شود
\فقره تصمیم معماری : محیطهای نظارتی از وجود خطا با خبر می‌شوند و به زیر سامانه‌ی مربوطه اطلاع می‌دهند. این نیازمندی برطرف نمی‌شود و محیط نظارتی در نظر گرفته نشده است. 
\end{itemize}
\زیرزیرقسمت{دسترسی}
\begin{itemize}
\فقره منبع : یکی از سرویس ها
\فقره محرک : خطایی رخ می‌دهد و باعث شکست سرویس می‌شود
\فقره محصول : سرویس مربوطه 
\فقره محیط : سامانه در عملکرد کاری معمولی است
\فقره پاسخ : استقرار جدیدی از سامانه برپا می شود
\فقره معیار پاسخ : استقرار جدید در کمتر از 30 ثانیه برپا می‌شود
\فقره تصمیم معماری : محیطهای نظارتی از وجود خطا با خبر می‌شوند و به زیر سامانه‌ی مربوطه اطلاع می‌دهند. این نیازمندی به طور کامل برطرف نمی‌شود و محیط نظارتی در نظر گرفته نشده است. اما استقرار سرویس جدید به دلیل استفاده از الگوی SOA به راحتی انجام می‌پذیرد.
\end{itemize}

\زیرزیرقسمت{امنیت}
\begin{itemize}
\فقره منبع : کاربر 
\فقره محرک : کاربر قصد دسترسی به آدرسی را دارد که مجاز نیست
\فقره محصول :مولفه‌های connection ، auth 
\فقره محیط : سامانه در عملکرد کاری معمولی است
\فقره پاسخ : اجازه دسترسی به کاربر داده نمی شود
\فقره معیار پاسخ : هیچگونه اطلاعاتی فاش نمی شود
\فقره تصمیم معماری : از مکانیزم مناسب جهت احراز هویت کاربران و تعیین سطح دسترسی استفاده می‌شود
\end{itemize}


\زیرزیرقسمت{امنیت}
\begin{itemize}
\فقره منبع : سیستم حمله کننده 
\فقره محرک : تلاش به منظور تغییر داده در پایگاه داده
\فقره محصول : پایگاه داده
\فقره محیط : سامانه در عملکرد کاری معمولی است
\فقره پاسخ : این تلاش ثبت می‌شود
\فقره معیار پاسخ : هیچگونه اطلاعاتی فاش نمی شود
\فقره تصمیم معماری : پورتهای دسترسی به پایگاه داده از خارج قابل دسترسی نیستند و فایروال از دسترسی به پایگاه داده از خارج از سازمان جلوگیری می‌کند. 
\end{itemize}


\زیرزیرقسمت{تغییر پذیری}
\begin{itemize}
\فقره منبع : تیم توسعه سیستم 
\فقره محرک : یک عملکرد به یک سرویس  اضافه می‌شود	
\فقره محصول : کدهای مربوط به سرویس
\فقره محیط : سامانه در حال توسعه است
\فقره پاسخ : تغییرات اعمال می‌شود سرویس جهت تست استقرار می‌یابد
\فقره معیار پاسخ : هر تغییر حداقل ۱ روز زمان می‌برد
\فقره تصمیم معماری : با در نظر گرفتن توسعه‌ی جداگانه‌ی سرویس ها و طراحی modularize مولفه‌های گوناگون تغییرات به آسانی قابل اعمال هستند. همچنین استفاده از سامانه‌های کنترل نسخه و CI کمک فراوانی به افزایش تغییرپذیری می‌کنند. 
\end{itemize}

\زیرزیرقسمت{قابلیت همکاری}

\begin{itemize}
\فقره منبع : سامانه‌ی حاوی کد رباتها
\فقره محرک : یک endpoint از سامانه صدا زده می‌شود	
\فقره محصول :مولفه‌های connection، bot
\فقره محیط : سامانه در حال عملکرد عادی است
\فقره پاسخ : درخواست به درستی پاسخ داده می‌شود
\فقره معیار پاسخ : پاسخ در کمتر از $/01$ ثانیه داده می‌شود
\فقره تصمیم معماری : ارتباط با سامانه با در نظر گرفتن endpoint های RESTful امکان پذیر خواهد بود 
\end{itemize}



